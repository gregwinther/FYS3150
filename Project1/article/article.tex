\documentclass[10pt, a4paper]{amsart}

\usepackage[]{graphicx}
\usepackage[]{hyperref}
\usepackage[]{physics}

\usepackage[T1]{fontenc}

\title[Solving the Poisson-equation in one dimension]{Solving the Poisson-equation in one dimension \\
  \hrulefill\small{ FYS3150: Computational Physics }\hrulefill}

\author[Svalheim \& Winther-Larsen]{Trygve Leithe Svalheim \\
   Sebastian G. Winther-Larsen \\
  \href{https://github.com/gregwinther/FYS3150/}{\texttt{github.com/gregwinther}}}

\begin{document}

\begin{titlepage}
\begin{abstract}
Lorem ipsum dolor sit amet, consectur adipiscing lit. Nullam ut lacus eget lorem...
\end{abstract}
\maketitle
\tableofcontents
\end{titlepage}

\section{Introduction}

\section{Theory}
\subsection{The Poisson Equation}

The Poisson equation is a classical equation from electromagnetism. The electrostatic potential $\Phi$ is generated by a localized charge distribution $\rho(\vb{r})$. In three dimensions the equation reads
\begin{equation}
\laplacian\Phi=-4\pi \rho(\vb{r})
\end{equation}
where $\laplacian$ is the Laplace operator. In three dimensions the Laplace operator can be expressed using spherical coordinates, but in this study I am assuming that $\Phi$ and $\rho$ are spherically symmetric, thus reducing the equation to a one-dimensional problem. only dependent on radius $r$.
\begin{equation}
\laplacian=\frac{1}{r^2}\frac{d}{dr}\left(r^2\frac{d\Phi}{dr} \right)
\end{equation}
By substituting $\Phi(r)=\phi(r)/r$ the Poisson equation is reduced to 
\begin{equation}
\frac{d^2\phi}{dr^2}=-4\pi r\rho(r)
\end{equation}
and by letting $\phi \rightarrow u$ and $r \rightarrow x$ one is left with the very simple equation
\begin{equation}
-u''(x)=f(x) \label{eq:2nd}
\end{equation}

The inhomogenous term $f$, or source term, is given by the charge distribution $\rho$ multiplied by $r$ and the constant $-4\pi$. In this study, however, the source term will be $f(x)=100e^{-10x}$ and the results can be compared to the analytical solution $u(x)=1-(1-e^{-10})x-e^{-10x}$. 


\subsection{Approximation of the Second Derivative}
In this study the one-dimensional Poisson equation will be solved with Dirichlet boundary conditions by rewriting it as a set of linear equations. The discretized approximation of $u$ is defined as $v_i$ with grid points $x_i=ih$, step size of $h=\frac{1}{n+1}$, in the interval $x_0=0$ to $x_{n+1}=1$ and with boundary conditions $v_0=v_n+1=0$. The interior solution $v_i \forall i \in {1,...,n}$ is to be found. The second order derivative is approximated with the three point formula such that equation \ref{eq:2nd} becomes
\begin{equation}
-\frac{v_{i+1}-2v_i+v{i-1}}{h²}=f_i \label{eq:2approx}
\end{equation}

By defining $\vb{f}=h^2f_i$ one can rewrite equation
\ref{eq:2approx} as $-v_{i+1}-2v_i+v_{i-1}=h^2f_i$. If we ignore the
end points, $i=0$ and $i=n+1$, this equation can be represented as a
matrix equation.
\begin{equation}
A\vb{v}=\vb{f}
\end{equation}
\begin{equation}
\begin{bmatrix}
2 & -1 & 0 & 0 & \cdots & 0 & 0 & 0 \\
-1 & 2 & -1 & 0 & \cdots & 0 & 0 & 0 \\
0 & -1 & 2 & -1 & \cdots & 0 & 0 & 0 \\ 
& & \vdots &  & \ddots &  & \vdots & \\
0 & 0 & 0 & 0 & \cdots & -1 & 2 & -1 \\
0 & 0 & 0 & 0 & \cdots & 0 & -1 & 2 
\end{bmatrix}
\begin{bmatrix}
v_1 \\
v_2 \\
v_3 \\
\vdots \\
v_{n-1} \\
v_n
\end{bmatrix}=
\begin{bmatrix}
f_1 \\
f_2 \\
f_3 \\
\vdots \\
f_{n-1} \\
f_n
\end{bmatrix}
\end{equation}

\section{Algorithms}
Two main methods are implemented and compared. The first method is
gaussian elimination of the tridiagonal matrix $A$, also known as the
\emph{Thomas Algorithm} \cite{thomasalgo}. This is a simplified form of Gaussian
elimination that can ve used to solve tridiagonal systems of
equations. The method is improved upon in order to take into account
the fact that the matrix we are dealing with has the same numbers
along the diagonals. The second method is the LU-decomposistion
method.

\subsection{Tridiagonal Matrix Algorithm}
Our tridiagonal system can be represented by 
\begin{equation}
a_ix_{i-1} + b_ix_i + c_ix_{i+1} = \vb{f}
\end{equation}
with $a_i=-1,b_i=2$ and $c_i=-1$, except for $a_1=0$ and $c_n=0$. Row
reducing a matrix will reveal how the algorithm functions. Limiting
the problem to four dimensions for easier reading and to save the
rainforest\footnote{Writing out a general case will take up more paper
space}.

\begin{equation}
\left[
\begin{array}{cccc|c}
b_1 & c_1 & 0 & 0 & f_1 \\
a_2 & b_2 & c_2 & 0  & f_2 \\
0 & a_3 & b_3 & c_3 & f_3 \\
0 & 0 & a_4 & b_4 & f_4
\end{array}
\right] \sim
\left[
\begin{array}{cccc|c}
1 & c_1/b_1 & 0 & 0 & b_1/b_1 \\
0 & b_2-\frac{c_1}{b_1}a_2 & c_2 & 0  & f_2-\frac{f_1}{b_1}a_2 \\
0 & a_3 & b_3 & c_3 & f_3 \\
0 & 0 & a_4 & b_4 & f_4
\end{array}
\right]
\end{equation}
Now let $\beta_1=b_1$ and $\beta_2=b_2-\frac{c_1}{b_1}a_2$. As new
elements start to appear in vector $\vb{f}$, right to the vertical bar
in the augmented matrix, they are also relabeled to $\tilde{f}_i$. For
example $\tilde{f}_1=f_1/\beta_1$. One more iteration will reveal the
pattern of the algorithm.

\begin{align}
\left[
\begin{array}{cccc|c}
1 & c_1/\beta_1 & 0 & 0 & \tilde{f}_1 \\
0 & 1 & c_2/\beta_2 & 0  & (f_2-\tilde{f}a_2)/\beta_2 \\
0 & a_3 & b_3 & c_3 & f_3 \\
0 & 0 & a_4 & b_4 & f_4
\end{array}
\right] &=
\left[
\begin{array}{cccc|c}
1 & c_1/\beta_1 & 0 & 0 & \tilde{f}_1 \\
0 & 1 & c_2/\beta_2 & 0  & \tilde{f}_2 \\
0 & a_3 & b_3 & c_3 & f_3 \\
0 & 0 & a_4 & b_4 & f_4
\end{array}
\right] \\ \sim 
\left[
\begin{array}{cccc|c}
1 & c_1/\beta_1 & 0 & 0 & \tilde{f}_1 \\
0 & 1 & c_2/\beta_2 & 0  & \tilde{f}_2 \\
0 & 0 & b_3-\frac{c_2}{\beta_2}a_3 & c_3 & f_3-\tilde{f}_2a_3 \\
0 & 0 & a_4 & b_4 & f_4
\end{array}
\right] &\sim 
\left[
\begin{array}{cccc|c}
1 & c_1/\beta_1 & 0 & 0 & \tilde{f}_1 \\
0 & 1 & c_2/\beta_2 & 0  & \tilde{f}_2 \\
0 & 0 & 1 & c_3/\beta_3 & (f_3-\tilde{f}_2a_3)/\beta_3 \\
0 & 0 & a_4 & b_4 & f_4
\end{array}\right] \\ 
\sim &\dots \sim
\left[
\begin{array}{cccc|c}
1 & c_1/\beta_1 & 0 & 0 & \tilde{f}_1 \\
0 & 1 & c_2/\beta_2 & 0  & \tilde{f}_2 \\
0 & 0 & 1 & c_3/\beta_3 & \tilde{f}_3 \\
0 & 0 & 0 & 1 & \tilde{f}_4
\end{array}
\right]
\end{align}

This pattern can be summarized quite elegantly by the following
difference equations.
\begin{equation}
\beta_i=b_i-\frac{c_{i-1}}{\beta_{i-1}}, \quad
\tilde{f}_i=(\beta_i-\tilde{f}_{i-1}a_i)/\beta_i, \quad i \in [2,n], \quad
\end{equation}

\begin{thebibliography}{10}
  \bibitem{thomasalgo}{Thomas, L.H. (1949), \emph{Elliptic
        Problems in Linear Differential Equations over a
        Network}. Watson Sci. Comput. Lab Report, Columbia University,
      New York.}
\end{thebibliography}

\end{document}

\documentclass[10pt, a4paper]{amsart}

\usepackage[]{graphicx}
\usepackage[]{hyperref}
\usepackage[]{physics}
\usepackage[]{listings}
\usepackage[utf8]{inputenc}

\title[Simulation of monetary transactions]{Monte Carlo simulation of monetary transactions: \\
\normalsize{A simple model for wealth distribution} \\
  \hrulefill\small{ FYS3150: Computational Physics }\hrulefill}

\author[Svalheim \& Winther-Larsen]{Trygve Leithe Svalheim \\
   Sebastian G. Winther-Larsen \\
  \href{https://github.com/gregwinther/FYS3150/}{\texttt{github.com/gregwinther}}}
  

\begin{document}

\begin{titlepage}
\begin{abstract}
Lorem ipsum dolor sit amet, consectetur adipiscing elit. In id neque elementum, accumsan ligula at, lobortis tortor. Duis elementum pellentesque purus, sit amet euismod diam facilisis volutpat. Nulla facilisi. Mauris quis felis ante. Aliquam ac velit sit amet velit porta condimentum iaculis ut quam. Phasellus pretium libero nec lectus placerat, ac consectetur sem faucibus. Maecenas dictum porta finibus.
\end{abstract}
\maketitle
\tableofcontents
\end{titlepage}

%% -----------------
%% INTRODUCTION
\section{Introduction}


%% -----------------
%% THEORY
\section{Theoretical Background}
\subsection{The simplest model for an economy}
Arguably, the most famous equation in macroeconomics is the ``autarky identity"\footnote{Any introductory text on macroeconomics will give a thorough elaboration, e.g. Gärtner.}
\begin{equation}
\label{eq:YCGI}
Y = C + G + I,
\end{equation}
where $Y$ is income, $C$ is consumption, $G$ is government spending and $I$ is investment. The best-know, but not necessarily the best, measure of income $Y$ is the Gross Domestic Product (GDP)\footnote{GDP = GNP (Gross National Product) in autarky.}. Consumption $C$ is the monetary value of all goods and services purchased in the private sector, while government spending $G$ is the consumption of the government. Finally, investment $I$ is the sum of private and public saving. 

Equation \ref{eq:YCGI} is autarkic because all interactions with other economies are excluded from the expression. There are no terms representing exports and captial inflow, for instance. We are dealing with a \emph{closed} economy, alternatively the entire world as a whole. Moreover, let's assume that the economy we are studying is a peaceful anarchy, without a governing authority of any sort, in effect setting $G = 0$. To begin with, we will also forgo the agents the ability to save such that the worth of every individual, or agent, in the economy must be spent at once. Equation \ref{eq:YCGI} is reduced to $Y = C$, everything one agent spends is the income of another.

\subsection{Monetary transactions}
To simulate monetary transactions in out model economy we expand employ the framework introduced in Patriarca et al.\cite{Patriarca}. We assume there are $N$ agents that exchange money in pairs ($i,j$). We assume also that all agents start with the same amount of money $m_0>0$. For every period an arbitrary pair of agents are picked at randomly and let them conduct business, id est a transaction takes place between them. Money is conserved during the transaction such that
\begin{equation}
\label{eq:wealthconserved}
m_i + m_j = m_i' + m_j',
\end{equation}
where the right-hand side is the agent $i$ and $j$'s updated wealth and the left-hand side represents the amount of money agents $i$ and $j$ had before the transaction. The exchange is done via a random reassignment factor $\epsilon$, such that
\begin{align}
m_i' &= \epsilon (m_i + m_j) \\
m_j' &= (1 - \epsilon) (m_i + m_j)
\end{align}
No agent will ever have negative wealth, that is $m \geq 0$. Moreover, because of the conservation law in equation \ref{eq:wealthconserved}, the system eventually reaches an equilibrium state given by a Gibbs distribution
\begin{equation}
\label{eq:gibbs}
w_m = \beta e^{-\beta m}, \quad \beta = \frac{1}{\ev{m}},
\end{equation}
where $\ev{m}$ is the expected wealth, for which the arithmetic mean is an unbiased estimator. This implies that after an equilibrium has been reached the majority of agents is left with lower wealth than they had initially and
the number of rich agents exponentially decrease.

\subsection{Transactions and savings}
We are can now expand upon the model by introducing a savings rate $\lambda$. The savings rate is defined as a fraction of an agent's wealth that does not partake in a transaction for every period. One can gather from the macroeconomic identity in equation \ref{eq:YCGI} that income must still be the same and the transaction law in equation \ref{eq:wealthconserved} still holds. The updated wealth of agents $i$ and $j$ after a transaction becomes
\begin{align}
m_i' &= \lambda m_i + \epsilon (1 - \lambda) (m_i + m_j) \\
m_j' &= \lambda m_j + (1- \epsilon) (1 - \lambda) (m_i + m_j)
\end{align}
one can rewrite these expressions to
\begin{align*}
m_i' &= m_i + \delta m \\
m_j' &= m_j - \delta m
\end{align*}
where
\begin{equation}
\delta m = (1 - \lambda)(\epsilon m_j - (1 - \epsilon)m_i)
\end{equation}

\subsection{Economic inequality and social friction}
``The first historical series of income distribution statistics became available with the publication in 1953 fo Kuznet's monumental Shares of Upper Income Groups in Income and Savings. Kuznet's series dealt with only one country (the United Stetes) over a period of thirty-five years (1913-1948)" (Pikkety \& Ganser, 2014, Kindle locations 276-280\cite{Piketty}). Ever since that time there has been wealth and income inequality to a greater or lesser degree. According to the PolitiFact the top 400 richest Americans in 2011 "[had] more wealth than half of all Americans combined"\cite{Moore}.

One of the possible reasons for continued wealth and income inequality that has risen in popularity the last few years is the process of wealth concentration. This is a process by which, under certain conditions, newly created wealth concentrates in the possession of already-wealthy individuals or entities. Those who already have wealth have the means to invest in newly created sources and structures of wealth. Piketty argues that the fundamental force for wealth divergence is the usually greater return of capital than economic growth (Piketty \& Ganser, 2014 p. 284 Table 12.2\cite{Piketty}).

To incorporate the social rigidities described above, we will make two further additions to the model. These additions build mostly on the work of Goswami and Sen\cite{GoswamiSen}. Firstly, we will make it more likely for two agents to conduct a transaction if they have similar wealth. Second, we will make it more likely for two agents to make a transaction if they have made a transaction before. We define a probability
\begin{equation}
\label{eq:rigidprob}
p_{ij} \propto \abs{m_i - m_j}^{-\alpha}(c_{ij}+1)^{\gamma},
\end{equation}
where the first factor $\abs{m_i-m_j}^{-\alpha}$ relates to wealth similarity and the second factor $(c_{ij}+1)^{\gamma}$ relates to previous transactions. A relatively large difference between $m_i$ and $m_j$ should translate to a lower probability. The strength of this effect is governed by the exponent $\alpha$, a larger value decreases the probability of a trade further. The variable $c_{ij}$ is simply the number of times agent $i$ and $j$ have conducted transactions in the past. The number $1$ is added in order to ensure that if they have not interacted earlier they can still interact. The strength of this effect is governed by the exponent $\gamma$.


%% -----------------
%% ALGORITHM
\section{Algorithm}

%% -----------------
%% RESULTS
\section{Results}

%% ----------------
%% DISCUSSION
\section{Discussion}

%% ---------------
%% CONCLUSION
\section{Summary Remarks}


\begin{thebibliography}{10}

\bibitem{Patriarca} Patriarca, M., Chakraborti, A., \& Kaski, K. (2004). 
	Gibbs versus non-Gibbs distributions in money dynamics. 
	\emph{Physica A: Statistical Mechanics and its Applications},
	340(1), 334-339.
	
\bibitem{Piketty} Piketty, T., \& Ganser, L. J. (2014).
	\emph{Capital in the twenty-first century}.
	
\bibitem{Moore} Moore, M., (March 7, 2011).
	The Forbes 400 vs. everybody else.
	\emph{\href{https://web.archive.org/web/20110309211959/http://www.michaelmoore.com/words/must-read/forbes-400-vs-everybody-else}{michaelmoore.com}}.
	Retrieved December 8. 2016.

\bibitem{GoswamiSen} Goswami, S., \& Sen, P. (2014).
	Agent based models for wealth distribution with preference in interaction.
	\emph{Physica A: Statistical Mechanics and its Applications},
	415, 514-524.

\end{thebibliography}

\end{document}